\appendix
\chapter{Основные аппроксимации и вывод системы прогностических уравнений модели ReMeDy}
\label{app:A}
Начнем с рассмотрения системы уравнений гидродинамики в невязком адиабатическом приближении в $z$-системе координат на $f$-плоскости:

\begin{subequations}\label{eq:eqn0}
\begin{align}
\frac{Du}{Dt}&=-\frac{1}{\rho}\pderiv{p}{x}+fv \label{eq:eqn1}\\
\frac{Dv}{Dt}&=-\frac{1}{\rho}\pderiv{p}{y}-fu \label{eq:eqn2}\\
\frac{Dw}{Dt}&=-\frac{1}{\rho}\pderiv{p}{z}-g \label{eq:eqn3}\\
\pderiv{u}{x}+\pderiv{v}{x}+\pderiv{w}{x}&=-\frac{Dln\rho}{Dt} \label{eq:eqn4}\\
\frac{Dln\theta}{Dt}&=0 \\
\intertext{\text{и уравнение состояния}}
p&=R\rho T, \label{eq:eqn5}\\
\intertext{\text{а потенциальная температура определяется как}}
\theta&=T\left(\frac{p_0}{p}\right)^\kappa. \label{eq:eqn6}
\end{align}
\end{subequations}

Далее в качестве вертикальной координаты примем
\begin{equation}
\sigma=\frac{p-p_{top}}{p_*}, p_*=p_{surf}-p_{top},
\end{equation}
которая превращается в традиционную $\sigma$-координату при $p_{top}=0$. В последующих выкладках примем, что $p_{top}$ есть константа. Производные в $\sigma$-системе коодинат соотносятся с производными в $z$-координатах следующим образом:
\begin{align}
\begin{split}
\left(\frac{\partial}{\partial x}\right)_z&=\left(\frac{\partial}{\partial x}\right)_{\sigma}-\left(\frac{\partial z}{\partial x}\right)_{\sigma} \left(\frac{\partial \sigma}{\partial z}\right) \left(\frac{\partial}{\partial \sigma}\right) \\
\frac{\partial}{\partial z}&=\frac{\partial\sigma}{\partial z}\frac{\partial}{\partial \sigma}
\end{split} \label{eq:z2sigma1}
\end{align}

Миллер и Уайт \citep{MillerWhite1984} получили систему уравнений в $p$-системе координат, применив ряд преобразований относительно связанного с фоновым состоянием параметром атмосферы, который имеет малые значения для широкого спектра атмосферных условий. Процедура может быть представлена в три этапа: \\
\begin{enumerate}
\item представление уравнений в новой системе координат, при это полагая $Dw/Dt$ параметром во всех уравнениях, кроме уравнения для вертикального ускорения;
\item определение фонового состояния как состояния гидростатического равновесия, в котором переменные зависят только от давления;
\item упрощение уравнений относительно фонового состояния с избавлением от членов высоких порядков.
\end{enumerate}

То есть фоновое состояние атмосферы определяется через переменные $$\phi_s(p), T_s(p)~or~\theta_s(p),$$
при чем
\begin{equation}
\frac{\partial\phi_s}{\partial\sigma}=\frac{\partial\phi_s}{\partial p}\frac{\partial p}{\partial\sigma}=-\frac{RT_s}{p}p_*
\end{equation}
Далее, следуя \citep{MillerWhite1984}, введем параметр
\begin{equation}
\epsilon=\frac{1}{g}\frac{Dw}{Dt},
\end{equation}
который является отношением вертикального ускорения к ускорению силы гравитации. Тогда уравнение для вертикальной скорости \eqref{eq:eqn3} может быть решено относительно $1/\rho$:
\begin{equation}
\frac{1}{\rho}=-\frac{g(1+\epsilon)}{\frac{\partial p}{\partial z}} \label{eq:vol}
\end{equation}

Подставляя последнее выражение в уравнения для горизонтальных компонент скорости \eqref{eq:eqn1} и \eqref{eq:eqn2} и используя \eqref{eq:z2sigma1}, получаем:
\begin{subequations}\label{eq:hmom0}
\begin{align}
\frac{Du}{Dt}&=\frac{g(1+\epsilon)}{\pderiv{p}{\sigma}\pderiv{\sigma}{z}}
\left[
\left(\pderiv{p}{x}\right)_\sigma-\pderiv{p}{\sigma}\pderiv{\sigma}{z}\left(\pderiv{z}{x}\right)_\sigma
\right]
+fv=g(1+\epsilon)
\left[\pderiv{\sigma}{p}\pderiv{z}{\sigma}\left(\pderiv{p}{x}\right)_\sigma-\left(\pderiv{z}{x}\right)_\sigma
\right]+fv=\nonumber\\
&=(1+\epsilon)
\left[
\frac{\sigma}{p_*}\left(\pderiv{p_*}{x}\right)_\sigma\pderiv{\phi}{\sigma}-\left(\pderiv{\phi}{x}\right)_\sigma
\right]+fv \nonumber \\
\intertext{or} \nonumber\\
\frac{Du}{Dt}&=(1+\epsilon)
\left[
-\left(\pderiv{\phi'}{x}\right)_\sigma+\pderiv{\phi'}{\sigma}\frac{\sigma}{p_*}\left(\pderiv{p_*}{x}\right)_\sigma
-\left(\pderiv{\phi_s}{x}\right)_\sigma+\pderiv{\phi_s}{\sigma}\frac{\sigma}{p_*}\left(\pderiv{p_*}{x}\right)_\sigma
\right]+fv \nonumber\\
\frac{Du}{Dt}&=(1+\epsilon)
\left[
-\left(\pderiv{\phi'}{x}\right)_\sigma+\pderiv{\phi'}{\sigma}\frac{\sigma}{p_*}\left(\pderiv{p_*}{x}\right)_\sigma
\right]+fv\label{eq:hmom1}
\intertext{and in a similar way:} \nonumber\\
\frac{Dv}{Dt}&=(1+\epsilon)
\left[
-\left(\frac{\partial\phi'}{\partial y}\right)_\sigma+\frac{\partial\phi'}{\partial\sigma}\frac{\sigma}{p_*}\left(\frac{\partial p_*}{\partial y}\right)_\sigma
\right]-fu\label{eq:hmom2}
\end{align}
\end{subequations}

Рассмотрим уравнение для вертикальной скорости \eqref{eq:eqn3}. Оно может быть переписано в виде
\begin{align}
g(1+\epsilon)&=-\frac{1}{\rho}\pderiv{p}{z} \\
\intertext{\text{или}} \nonumber\\
g(1+\epsilon)&=-\frac{1}{\rho}p_*\pderiv{\sigma}{z} \nonumber\\
(1+\epsilon)\left(\pderiv{\phi'}{\sigma}+\pderiv{\phi_s}{\sigma}\right)&=-\frac{1}{\rho}p_*\nonumber\\
\intertext{or, using the hydrostatic relation for the reference state:} \nonumber \\
(1+\epsilon)\left(\pderiv{\phi'}{\sigma}-p_*\frac{1}{\rho_s}\right)&=-\frac{1}{\rho}p_*\nonumber\\
-\epsilon p_*\frac{1}{\rho_s}-p_*\frac{1}{\rho_s}+(1+\epsilon)\pderiv{\phi'}{\sigma}&=-\frac{1}{\rho}p_*\nonumber\\
\epsilon p_*\frac{1}{\rho_s}&=(1+\epsilon)\pderiv{\phi'}{\sigma}+p_*\left(\frac{1}{\rho}-\frac{1}{\rho_s}\right)
\end{align}
Подставим $\epsilon$, согласно его определению, но только в первое слагаемое:
\begin{align}
\frac{Dw}{Dt}&=(1+\epsilon)\pderiv{\phi'}{\sigma}\frac{g\rho_s}{p_*}-g\frac{(\rho-\rho_s)}{\rho}\nonumber\\
\intertext{\text{и при} $\frac{\rho'}{\rho_s+\rho'}\approx\frac{\rho'}{\rho_s}$} \nonumber\\
\frac{Dw}{Dt}&=(1+\epsilon)\frac{g\rho_s}{p_*}\pderiv{\phi'}{\sigma}-g\frac{\rho'}{\rho_s} \label{eq:dwdt_rho}
\end{align}
Последнее слагаемое в правой части ур. \eqref{eq:dwdt_rho} может быть далее представлено как $g\frac{\theta'}{\theta_s}$.

С другой стороны, по определению вертикальное ускорение выглядит как
\begin{align}
\frac{Dw}{Dt}&=\frac{1}{g}\frac{D}{Dt}\left(\frac{D\phi'}{Dt}+\frac{D\phi_s}{Dt}\right)
\end{align}
и
\begin{align}
\frac{D\phi_s}{Dt}&=\pderiv{\phi_s}{p}\frac{D\phi_s}{Dt}=-\frac{RT_s}{p}\omega=-\frac{\omega}{\rho_s}.
\end{align}
Используя последнее, запишем:
\begin{align}
\tilde{w}&=\frac{1}{g}\frac{D\phi_s}{Dt}=-\frac{\omega}{\rho_sg} \\
\tilde{w}&=-\frac{1}{\rho_sg}\frac{Dp}{Dt}=-\frac{1}{S}(\dot{\sigma}+\frac{\sigma}{p_*}\dot{p_*}),
\end{align}
а окончательный вид уравнения для вертикальной скорости имеет вид
\begin{equation}
\frac{D}{Dt}\left[\frac{1}{g}\frac{D\phi'}{Dt}+\tilde{w}\right]=(1+\epsilon)S\pderiv{\phi'}{\sigma}+g\frac{\theta'}{\theta_s},
\end{equation}
где 
\begin{equation}
S=\frac{gp}{RT_sp_*}.
\end{equation}
Заметим, что 
\begin{equation}
-S\pderiv{}{\sigma}\approx\pderiv{}{z} \label{eq:dz}
\end{equation}

Преобразование уравнения неразрывности можно начать с рассмотрения уравнения, обобщенного на случай произвольной вертикальной координаты.
\begin{equation}
\frac{Dln\rho}{Dt}+\pderiv{u}{x}+\pderiv{v}{y}+\pderiv{\dot{\eta}}{\eta}+\frac{D}{Dt}\left(ln\left(\pderiv{z}{\eta}\right)\right)=0
\end{equation}
Имея в качестве вертикальной координаты $\sigma$-координату, получим уравнение неразрывности в системе координат модели:
\begin{equation}
\frac{Dln\rho}{Dt}+\frac{D}{Dt}\left(ln\left(\pderiv{z}{\sigma}\right)\right)+\pderiv{u}{x}+\pderiv{v}{y}+\pderiv{\dot{\sigma}}{\sigma}=0
\end{equation}
Используя опять уравнение \eqref{eq:vol}, переписанное в виде
\begin{equation}
\frac{D}{Dt}\left(ln\left(\pderiv{z}{\sigma}\right)\right)=\frac{Dlnp_*}{dt}-\frac{Dln\rho}{Dt}-\frac{Dln(1+\epsilon)}{Dt},
\end{equation}
получаем
\begin{equation}
\frac{Dlnp_*}{Dt}+\pderiv{u}{x}+\pderiv{v}{y}+\pderiv{\dot{\sigma}}{\sigma}=\frac{Dln(1+\epsilon)}{Dt}.
\end{equation}

Уравнение притока тепла может быть получено следующим образом. Сначала выпишем его в изобарических координатах:
\begin{equation}
\frac{D\theta}{Dt}=\frac{D\theta'}{Dt}+\frac{D\theta_s}{Dt}=\frac{D\theta'}{Dt}+\omega\pderiv{\theta_s}{p}=0.
\end{equation}
Далее, используя определение $\tilde{w}$ и свойства фонового состояния, получаем:
\begin{equation}
\frac{D\theta'}{Dt}-S\tilde{w}\pderiv{\theta_s}{\sigma}=0
\end{equation}
%which (neglecting the last term) can be written in a more familiar form, noting that:
%\begin{equation}
%-S\pderiv{}{\sigma}\approx\pderiv{}{z}
%\end{equation}
%and defining:
%\begin{equation}
%N^2=-S_v\frac{g}{\theta_s}\frac{\partial\theta_s}{\partial\sigma} \label{bfreq}
%\end{equation}
%leading to
%\begin{equation}
%\frac{g}{\theta_s}\frac{D\theta'}{Dt}+N^2\tilde{w}=0.
%\end{equation}

Система уравнений записана теперь в форме, удобной для упрощения, которое производится просто путем отбрасывания всех слагаемых, в которые входит параметр $\epsilon$ и величину $1/g\left(D^2\phi'/Dt^2\right)$. Миллер и Уайт обосновывают это приближение путем разложения членов уравнений в степенные ряды от малого параметра $\alpha$, зависящего лишь от фонового состояния. После чего отбрасываются члены высокого порядка и выводится система уравнений, подходящая для численного интегрирования. Условие, при котором описанная аппроксимация верна, следующее:
\begin{equation}
\alpha=\frac{\Delta\theta_s}{\theta_s}=\frac{p_0}{\theta_s}\pderiv{\theta_s}{p}\ll1,
\end{equation}
что всегда выполняется в тропосфере.