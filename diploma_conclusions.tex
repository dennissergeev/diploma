%\documentclass[12pt,a4paper]{report}
%%
% PACKAGES AND STYLES
%

% 
% Language and encoding
%
%\usepackage{mathtext}
%\usepackage[T2A]{fontenc}
\usepackage[utf8]{inputenc}
\usepackage[english,russian]{babel}
\usepackage{mmap}

%
% Colors
%
\usepackage[usenames]{color}
\usepackage{color}
\usepackage{colortbl}

%
% Symbols
%
\usepackage{amssymb}
\usepackage{MnSymbol}

%
% Units
%
%\usepackage[binary-units=true]{siunitx}
\newcommand{\km}{\mathrm{~\text{км}}}
\newcommand{\m}{\mathrm{~\text{м}}}
\newcommand{\s}{\mathrm{~\text{с}}}
\newcommand{\mps}{\m \s ^{-1}}
\newcommand{\pers}{\s ^{-1}}
\newcommand{\K}{\mathrm{~\text{К}}}
\newcommand{\Kpkm}{\K\km ^{-1}}
\newcommand{\hpa}{\mathrm{~\text{гПа}}}
\newcommand{\J}{\mathrm{~\text{Дж}}}
\newcommand{\Jpm}{\J \m ^{-3}}

%
% Paper size and margins
%
\usepackage{vmargin}
\setmarginsrb{2.5cm}{1cm}{2.5cm}{2cm}{0cm}{0cm}{0cm}{1.5cm}

%
% Page style
%
\usepackage{fancyhdr}
\setlength{\headheight}{16pt}
\newcommand{\changefont}{%
    \fontsize{9}{11}\selectfont
}
\fancyhf{}
\fancyhead[RO]{\changefont \slshape \rightmark} %section
\fancyhead[LO]{\changefont \slshape \leftmark} %chapter
\fancyfoot[C]{\changefont \thepage} %footer
\setlength{\headsep}{0.2in}
\pagestyle{fancy}

%
% Indenting
%
\usepackage{indentfirst}
\setlength{\parindent}{1cm}
\setlength{\parskip}{0.5cm}

%
% References
%
\usepackage{natbib}

%
% Hyperlinks
%
\usepackage[linktocpage=true,plainpages=false,pdfpagelabels=false]{hyperref}
\definecolor{linkcolor}{rgb}{0.1,0,0.9}
\definecolor{citecolor}{rgb}{0,0,0.9}
\definecolor{urlcolor}{rgb}{0,0,1}
\hypersetup{
    colorlinks, linkcolor={linkcolor},
    citecolor={citecolor}, urlcolor={urlcolor}
}

\bibliographystyle{plainnat}
% Bibliography: set article volume number in bold font
%\DeclareFieldFormat
%  [article]
%  {volume}{\textbf{#1}}
%\renewcommand\nameyeardelim{, }

%\usepackage{showkeys} % show labels

\newcommand{\figref}[1]{\mbox{Figure~\ref{#1}}}
\newcommand{\tabref}[1]{\mbox{Table~\ref{#1}}}
\newcommand{\secref}[1]{\mbox{Section~\ref{#1}}}
\newcommand{\chpref}[1]{\mbox{Chapter~\ref{#1}}}
\newcommand{\appref}[1]{\mbox{Appendix~\ref{#1}}}
\newcommand{\eqnref}[1]{\mbox{Eq.~(\ref{#1})}}
\newcommand{\listref}[1]{\mbox{Listing~(\ref{#1})}}

%
% Lists
%
\usepackage[shortlabels]{enumitem}

\newenvironment{sqlist}[1][\enskip$\filledsquare$]
        {\begin{itemize}[#1]}
        {\end{itemize}}

% New math commands
% differential d, from http://tex.stackexchange.com/a/60546/586
\newcommand*\diff{\mathop{}\!\mathrm{d}}
\newcommand\mean[1]{\overline{#1}}
\newcommand{\PDt}[2]{\partial #1/\partial #2}

%
% Tables
%
\usepackage{booktabs}
%\usepackage{tabularx}
\usepackage{tabu}
\usepackage{longtable}

%
% Figures
%
\usepackage{wrapfig}
\usepackage[font=small,textfont=it,labelfont=bf]{caption}
%\captionsetup[figure]{labelfont=bf}
\usepackage{tikz}
\usepackage{pgfplots}
\usetikzlibrary{calc}

%
% Equations
%
\usepackage{cool}

%\begin{document}
\chapter*{Заключение}
\addcontentsline{toc}{section}{Заключение}
\markboth{Заключение}{ }

Настоящая дипломная работа основывается на современных подходах к изучению полярных мезоциклонов и представляет собой анализ идеализированных численных экспериментов с использованием строгой энергетической диагностики. Подводя итоги, выделим основные результаты исследования:

\begin{enumerate}

\item Трехмерная негидростатическая модель ReMeDy была использована для воспроизведения формирования полярного мезоциклона конвективного характера в атмосфере при наличии или отсутствии эффектов выделения скрытой теплоты конденсации водяного пара. Была подробно рассмотрена структура осесимметричного циклонического возмущения, причиной возникновения которого являлась положительная аномалия потенциальной температуры воздуха вблизи поверхности. Динамика атмосферных движений была оценена с точки зрения уравнений баланса кинетической и доступной потенциальной энергии, а также в аспекте бюджета завихренности в нижней тропосфере.

\item С использованием общепринятой методики был сопоставлен вклад основных видов конвективной неустойчивости в эволюцию вихря. Установлено, что определяющим механизмом интенсификации полярного мезоциклона в проведенных экспериментах являлось взаимодействие атмосферы и океана в приводном слое атмосферы, описанное в теории WISHE.

\item По результатам нескольких серий численных экспериментов была выявлена зависимость интенсивности полярного вихря от некоторых параметров фоновой атмосферы и от характеристик начальной аномалии тепла. Чувствительность циклона к различным факторам была оценена количественно как отношение изменения интенсивности барической тенденции к изменению того или иного параметра. К параметрам атмосферы, рассмотренным в данном исследовании, относятся разность температуры воздуха и поверхности океана, влагосодержание воздуха, вертикальный градиент температуры, скорость фонового потока, а также амплитуда начального температурного возмущения. Чувствительность к начальному влагосодержанию оказалась довольно низкой, так как даже при пониженной относительной влажности в начале эксперимента в нижнем слое атмосферы быстро достигалось состояние насыщения. Средние показатели чувствительности к перечисленным факторам приведены в табл. \ref{tab:conclusions}.

\renewcommand{\arraystretch}{2}
\begin{table}
\centering
\caption{Чувствительность интенсивности вихря к параметрам фоновой атмосферы и амплитуде начальной аномалии температуры (значения на единицу изменения параметра).}
\label{tab:conclusions}
\small
\begin{tabu} to 0.8\linewidth {X[l]X[l]}
\toprule
Параметр, $P$ & Чувствительность, $\delta I/\delta P$ \\
\midrule
Уменьшение частоты Брента-Вяйсяля & $-1.03\hpa/\text{час}~(\pers)^{-1}$ \\
Увеличение разности температуры 'вода-воздух' & $-1.38\times 10^{-2}\hpa/\text{час}~\K ^{-1}$ \\
Увеличение скорости фонового потока & $-1.77\times 10^{-1}\hpa/\text{час}~(\mps)^{-1}$ \\
Увеличение начальной амплитуды & $-3.23\times 10^{-1}\hpa/\text{час}~\K ^{-1}$ \\
\bottomrule
\end{tabu}
\end{table}

\item На основе нескольких параметризаций потоков тепла и импульса с поверхности был оценен эффект использования схем перемешивания в приводном слое атмосферы. Было установлено, что наибольшее отличие вносит включение свободной конвекции, что отражается на циркуляции во всей расчетной области и снижает скорость роста центрального возмущения.

\item В модель ReMeDy была включена и успешно протестирована параметризация подсеточного перемешивания в пограничном слое атмосферы. Использование нелокального замыкания, учитывающего эффект противоградиентного переноса тепла в нижней тропосфере, показало на качественном уровне более близкое к данным наблюдений воспроизведение конвективного пограничного слоя атмосферы. Это сильно отразилось на динамике циклонического возмущения: при использовании улучшенного подсеточного перемешивание при прочих равных условиях мезоциклон развивался менее интенсивно, так, в 'сухой' атмосфере интенсификации вихря до ураганной силы не происходило.

\item Большое количество оценочных экспериментов требует дальнейшего совершенствования мезомасштабной численной модели ReMeDy. В ходе работы над представленным исследованием улучшения модели велись по таким направлениям, как:
	\begin{sqlist} 
		\item сбалансированная инициализация метеорологических полей для избежания генерации ложных волн;
		\item параметризация подсеточного турбулентного перемешивания;
		\item включение периодических граничных условий для возможности моделирования мезомасштабных циркуляций в зональных потоках; 
		\item способы инициализации начального возмущения;
		\item постпроцессинг данных моделирования.
	\end{sqlist}
	
Одним из факторов, к которым эксперименты оказались наиболее чувствительны, являются граничные условия. Вычислительная мода часто возникала на границах области, что возможно говорит о необходимости доработки горизонтальных и вертикальных численных фильтров.

\item Необходимы дальнейшие исследования чувствительности динамики мезоциклона к воспроизведению пограничного слоя и подсеточному перемешиванию атмосферы. Проблема правильного подбора схемы подсеточного перемешивания актуальна как для современных мезомасштабных моделей, используемых в оперативной практике, так и для глобальных моделей общей циркуляции атмосферы.

\end{enumerate}

%\end{document}