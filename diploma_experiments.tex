\documentclass[12pt,a4paper]{report}

%
% PACKAGES AND STYLES
%

% 
% Language and encoding
%
%\usepackage{mathtext}
%\usepackage[T2A]{fontenc}
\usepackage[utf8]{inputenc}
\usepackage[english,russian]{babel}
\usepackage{mmap}

%
% Colors
%
\usepackage[usenames]{color}
\usepackage{color}
\usepackage{colortbl}

%
% Symbols
%
\usepackage{amssymb}
\usepackage{MnSymbol}

%
% Units
%
%\usepackage[binary-units=true]{siunitx}
\newcommand{\km}{\mathrm{~\text{км}}}
\newcommand{\m}{\mathrm{~\text{м}}}
\newcommand{\s}{\mathrm{~\text{с}}}
\newcommand{\mps}{\m \s ^{-1}}
\newcommand{\pers}{\s ^{-1}}
\newcommand{\K}{\mathrm{~\text{К}}}
\newcommand{\Kpkm}{\K\km ^{-1}}
\newcommand{\hpa}{\mathrm{~\text{гПа}}}
\newcommand{\J}{\mathrm{~\text{Дж}}}
\newcommand{\Jpm}{\J \m ^{-3}}

%
% Paper size and margins
%
\usepackage{vmargin}
\setmarginsrb{2.5cm}{1cm}{2.5cm}{2cm}{0cm}{0cm}{0cm}{1.5cm}

%
% Page style
%
\usepackage{fancyhdr}
\setlength{\headheight}{16pt}
\newcommand{\changefont}{%
    \fontsize{9}{11}\selectfont
}
\fancyhf{}
\fancyhead[RO]{\changefont \slshape \rightmark} %section
\fancyhead[LO]{\changefont \slshape \leftmark} %chapter
\fancyfoot[C]{\changefont \thepage} %footer
\setlength{\headsep}{0.2in}
\pagestyle{fancy}

%
% Indenting
%
\usepackage{indentfirst}
\setlength{\parindent}{1cm}
\setlength{\parskip}{0.5cm}

%
% References
%
\usepackage{natbib}

%
% Hyperlinks
%
\usepackage[linktocpage=true,plainpages=false,pdfpagelabels=false]{hyperref}
\definecolor{linkcolor}{rgb}{0.1,0,0.9}
\definecolor{citecolor}{rgb}{0,0,0.9}
\definecolor{urlcolor}{rgb}{0,0,1}
\hypersetup{
    colorlinks, linkcolor={linkcolor},
    citecolor={citecolor}, urlcolor={urlcolor}
}

\bibliographystyle{plainnat}
% Bibliography: set article volume number in bold font
%\DeclareFieldFormat
%  [article]
%  {volume}{\textbf{#1}}
%\renewcommand\nameyeardelim{, }

%\usepackage{showkeys} % show labels

\newcommand{\figref}[1]{\mbox{Figure~\ref{#1}}}
\newcommand{\tabref}[1]{\mbox{Table~\ref{#1}}}
\newcommand{\secref}[1]{\mbox{Section~\ref{#1}}}
\newcommand{\chpref}[1]{\mbox{Chapter~\ref{#1}}}
\newcommand{\appref}[1]{\mbox{Appendix~\ref{#1}}}
\newcommand{\eqnref}[1]{\mbox{Eq.~(\ref{#1})}}
\newcommand{\listref}[1]{\mbox{Listing~(\ref{#1})}}

%
% Lists
%
\usepackage[shortlabels]{enumitem}

\newenvironment{sqlist}[1][\enskip$\filledsquare$]
        {\begin{itemize}[#1]}
        {\end{itemize}}

% New math commands
% differential d, from http://tex.stackexchange.com/a/60546/586
\newcommand*\diff{\mathop{}\!\mathrm{d}}
\newcommand\mean[1]{\overline{#1}}
\newcommand{\PDt}[2]{\partial #1/\partial #2}

%
% Tables
%
\usepackage{booktabs}
%\usepackage{tabularx}
\usepackage{tabu}
\usepackage{longtable}

%
% Figures
%
\usepackage{wrapfig}
\usepackage[font=small,textfont=it,labelfont=bf]{caption}
%\captionsetup[figure]{labelfont=bf}
\usepackage{tikz}
\usepackage{pgfplots}
\usetikzlibrary{calc}

%
% Equations
%
\usepackage{cool}


\begin{document}
\setcounter{chapter}{3}
\chapter{Численные эксперименты}
\section{Мезомасштабная модель ReMeDy}
\subsection{Система уравнений (вывод?)}
\subsection{Усовершенствования в модели}
\begin{sqlist}
\item Периодические гр. усл.
\item Инициализация модели
\item Существующие проблемы использования модели
\end{sqlist}

\subsection{Конечно-разностная сетка модели}

\section{Постановка экспериментов}
Данное исследование основывается на идеализированных численных экспериментах, позволяющих улучшить наше понимание динамики полярных мезоциклонов. Преимущества идеализированных экспериментов были приведены в первом разделе, а здесь обосновывается постановка задачи применительно к генерации мезоциклонов.
На основе предшествующих работ, как теоретических, так и посвященных анализу данных наблюдений за полярными вихрями, параметры атмосферы в модели были заданы относительно близкими к воздушным массам морских областей высоких широт. Зимой в этих районах нередко наблюдаются холодные вторжения со льда на поверхность открытой воды, в результате чего разность температуры воздуха и воды  достигает десятков градусов.

Этап проведения экспериментов был начат с подбора вертикальных профилей температуры, скорости ветра и удельной влажности воздуха. Затем какой-либо один ключевой параметр в начальных условиях варьировался, а остальные параметры оставались равными «контрольным». На этом принципе построены остальные эксперименты, называемые далее оценочными.

После получения результатов в виде трехмерных и двумерных полей метеовеличин, а также таких интегральных показателей, как кинетическая энергия, планировалось сопоставить скорости роста возмущений, барическую тенденцию в центре вихря, временной ход максимальной скорости ветра и завихренности, а также другие характеристики эволюции мезоциклонов. Помимо этого, была проведена энергетическая диагностика мезомасштабных движений в области расчетов, о разработке которой рассказано ниже (раздел ???).

Расчетная область имела размеры  по горизонтали  $1000\km \times$ 
%$ 1000~\text{км}$ и $10.5~\text{км}$ 
%по вертикали (высота верхней границы области оправдана для моделирования тропосферных процессов приполярных широт). В нескольких сериях экспериментов по техническим причинам горизонтальные размеры области интегрирования составили $380~\text{км} \times 380~\text{км}$ и $1200~\text{км} \times 1200~\text{км}$, однако это почти не повлияло на динамику экспериментов.


\end{document}