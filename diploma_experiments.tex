\documentclass[12pt,a4paper]{report}

%
% PACKAGES AND STYLES
%

% 
% Language and encoding
%
%\usepackage{mathtext}
%\usepackage[T2A]{fontenc}
\usepackage[utf8]{inputenc}
\usepackage[english,russian]{babel}
\usepackage{mmap}

%
% Colors
%
\usepackage[usenames]{color}
\usepackage{color}
\usepackage{colortbl}

%
% Symbols
%
\usepackage{amssymb}
\usepackage{MnSymbol}

%
% Units
%
%\usepackage[binary-units=true]{siunitx}
\newcommand{\km}{\mathrm{~\text{км}}}
\newcommand{\m}{\mathrm{~\text{м}}}
\newcommand{\s}{\mathrm{~\text{с}}}
\newcommand{\mps}{\m \s ^{-1}}
\newcommand{\pers}{\s ^{-1}}
\newcommand{\K}{\mathrm{~\text{К}}}
\newcommand{\Kpkm}{\K\km ^{-1}}
\newcommand{\hpa}{\mathrm{~\text{гПа}}}
\newcommand{\J}{\mathrm{~\text{Дж}}}
\newcommand{\Jpm}{\J \m ^{-3}}

%
% Paper size and margins
%
\usepackage{vmargin}
\setmarginsrb{2.5cm}{1cm}{2.5cm}{2cm}{0cm}{0cm}{0cm}{1.5cm}

%
% Page style
%
\usepackage{fancyhdr}
\setlength{\headheight}{16pt}
\newcommand{\changefont}{%
    \fontsize{9}{11}\selectfont
}
\fancyhf{}
\fancyhead[RO]{\changefont \slshape \rightmark} %section
\fancyhead[LO]{\changefont \slshape \leftmark} %chapter
\fancyfoot[C]{\changefont \thepage} %footer
\setlength{\headsep}{0.2in}
\pagestyle{fancy}

%
% Indenting
%
\usepackage{indentfirst}
\setlength{\parindent}{1cm}
\setlength{\parskip}{0.5cm}

%
% References
%
\usepackage{natbib}

%
% Hyperlinks
%
\usepackage[linktocpage=true,plainpages=false,pdfpagelabels=false]{hyperref}
\definecolor{linkcolor}{rgb}{0.1,0,0.9}
\definecolor{citecolor}{rgb}{0,0,0.9}
\definecolor{urlcolor}{rgb}{0,0,1}
\hypersetup{
    colorlinks, linkcolor={linkcolor},
    citecolor={citecolor}, urlcolor={urlcolor}
}

\bibliographystyle{plainnat}
% Bibliography: set article volume number in bold font
%\DeclareFieldFormat
%  [article]
%  {volume}{\textbf{#1}}
%\renewcommand\nameyeardelim{, }

%\usepackage{showkeys} % show labels

\newcommand{\figref}[1]{\mbox{Figure~\ref{#1}}}
\newcommand{\tabref}[1]{\mbox{Table~\ref{#1}}}
\newcommand{\secref}[1]{\mbox{Section~\ref{#1}}}
\newcommand{\chpref}[1]{\mbox{Chapter~\ref{#1}}}
\newcommand{\appref}[1]{\mbox{Appendix~\ref{#1}}}
\newcommand{\eqnref}[1]{\mbox{Eq.~(\ref{#1})}}
\newcommand{\listref}[1]{\mbox{Listing~(\ref{#1})}}

%
% Lists
%
\usepackage[shortlabels]{enumitem}

\newenvironment{sqlist}[1][\enskip$\filledsquare$]
        {\begin{itemize}[#1]}
        {\end{itemize}}

% New math commands
% differential d, from http://tex.stackexchange.com/a/60546/586
\newcommand*\diff{\mathop{}\!\mathrm{d}}
\newcommand\mean[1]{\overline{#1}}
\newcommand{\PDt}[2]{\partial #1/\partial #2}

%
% Tables
%
\usepackage{booktabs}
%\usepackage{tabularx}
\usepackage{tabu}
\usepackage{longtable}

%
% Figures
%
\usepackage{wrapfig}
\usepackage[font=small,textfont=it,labelfont=bf]{caption}
%\captionsetup[figure]{labelfont=bf}
\usepackage{tikz}
\usepackage{pgfplots}
\usetikzlibrary{calc}

%
% Equations
%
\usepackage{cool}


\begin{document}
\setcounter{chapter}{3}
\chapter{Численные эксперименты}
\section{Мезомасштабная модель ReMeDy}
\subsection{Система уравнений (вывод?)}
\subsection{Усовершенствования в модели}
\begin{sqlist}
\item Периодические гр. усл.
\item Инициализация модели
\item Существующие проблемы использования модели
\end{sqlist}

\subsection{Конечно-разностная сетка модели}

\section{Постановка экспериментов}
\label{sec:expsetup}
Данное исследование основывается на идеализированных численных экспериментах, позволяющих улучшить наше понимание динамики полярных мезоциклонов. Преимущества идеализированных экспериментов были приведены в первом разделе, а здесь обосновывается постановка задачи применительно к генерации мезоциклонов.
На основе предшествующих работ, как теоретических, так и посвященных анализу данных наблюдений за полярными вихрями, параметры атмосферы в модели были заданы относительно близкими к воздушным массам морских областей высоких широт. Зимой в этих районах нередко наблюдаются холодные вторжения со льда на поверхность открытой воды, в результате чего разность температуры воздуха и воды  достигает десятков градусов.

Этап проведения экспериментов был начат с подбора вертикальных профилей температуры, скорости ветра и удельной влажности воздуха. Затем какой-либо один ключевой параметр в начальных условиях варьировался, а остальные параметры оставались равными «контрольным». На этом принципе построены остальные эксперименты, называемые далее оценочными.

После получения результатов в виде трехмерных и двумерных полей метеовеличин, а также таких интегральных показателей, как кинетическая энергия, планировалось сопоставить скорости роста возмущений, барическую тенденцию в центре вихря, временной ход максимальной скорости ветра и завихренности, а также другие характеристики эволюции мезоциклонов. Помимо этого, была проведена энергетическая диагностика мезомасштабных движений в области расчетов, о разработке которой рассказано ниже (раздел ???).

Расчетная область имела размеры  по горизонтали  $1000\km \times 1000\km$ и $10.5\km$ по вертикали (высота верхней границы области оправдана для моделирования тропосферных процессов приполярных широт). В нескольких сериях экспериментов по техническим причинам горизонтальные размеры области интегрирования составили $380\km \times 380\km$ и $10.5\km$ и $1200\km \times 1200\km$ и $10.5\km$, однако это почти не повлияло на динамику экспериментов.

\begin{table}
\centering
\caption{Параметры сетки модели}
\label{tab:modelgrid}
\begin{tabular}{ll}
\toprule
Шаг по времени & $5\s$ \\
Шаг сетки по горизонтали & $10000\m$ \\
Шаг сетки по вертикали & от $30$ до $1000\m$ (30 уровней) \\
Размер области интегрирования	& $1000\km \times 1000\km \times 10.5\km$ \\
\bottomrule
\end{tabular}
\end{table}

Численные эксперименты проводились на сетке с разрешением $10\km$ (\ref{tab:modelgrid}) и с шагом по времени, равным $5\s$. Достаточно грубое для современных мезомасштабных моделей разрешение было выбрано с целью экономии вычислительных ресурсов. В планах дальнейшего исследования планируется уменьшение шага сетки по горизонтали, что, конечно, окажет положительный эффект на воспроизведение мезоциклонов, а именно вклад конвекции в их динамику. Влияние пространственного шага атмосферной модели в контексте изучения полярных мезоциклонов затрагивалось в работах \citep{YanaseNiino2005} и \citep{McInnesEtAl2011}. Авторы последней статьи доказывают улучшение прогноза полярных вихрей на примере Норвежского и Баренцева морей, демонстрируя чувствительность к начальным условиям при том или ином шаге сетки, а также важность согласованности разрешения модели при экспериментах с вложенными сетками.

\subsection{Постановка контрольного эксперимента}
\label{sec:expsetup:ctrl}
Начнем с описания начальных условий для контрольного эксперимента (здесь и далее эксперимент "CTRL"). Фоновое поле ветра задавалось равным нулю, то есть атмосфера находилась в покое. Такое состояние несвойственно районам холодных арктических вторжений или бароклинным зонам высоких широт. Тем не менее, отсутствие фонового потока позволяет на идеализированном примере подробно рассмотреть динамику и цикл обратных связей в растущем вихре. Данный эксперимент можно в первом приближении интерпретировать модель развития вихря при слабом влиянии крупномасштабных синоптических условий.

Приземное (приводное) значение температуры воздуха составляет $251\K$. Вертикальный градиент фоновой температуры постоянен и составляет $2\Kpkm$, горизонтально поле температуры однородно. Такой выбор сделан в соответствии с несколькими сериями наблюдений, свидетельствовавшими о слабой устойчивости в слое $5-7\km$ и сильной устойчивости в верхних слоях \citep{EmanuelRotunno1989}. Профиль относительной влажности воздуха задавался линейно меняющимся от $80\%$ на нижнем модельном уровне и до $0\%$ на $10\km$. В контрольном эксперименте процессы конденсации и выделения скрытого тепла в модели были выключены, и удельная влажность воздуха играла незначительную роль, внося вклад лишь в слагаемое плавучести \ref{eq:progn3}.
Температура поверхности воды равнялась $283\K$ (что находится в пределах наблюдаемых значений при развитии полярных мезомасштабных вихрей, \citep{ForbesLottes1985}), температура вблизи дна составляла $277\K$. Глубина водоема равнялась $50\m$, что оправдано на временных масштабах (3 сут.) проводимых экспериментах. Кроме того, увеличение глубины и вертикального разрешения модели LAKE ощутимо замедляет проведение расчетов.
В базовом эксперименте расчеты начинаются с задания искусственного возмущения поля температуры. Характеристики термической аномалии и принцип наложения описан в разделе \ref{sec:initanom}.

\subsection{Инициализация аномалии}
\label{sec:initanom}
В целом ряде работ (напр., \citep{Adakudlu2012}), основанных на идеализированных экспериментах, начальное возмущение метеорологических полей задавалось в виде осесимметричной аномалии ветра, температуры и давления, которая в дальнейшем развивалась в полярных мезоциклон. В качестве возмущения часто принимался либо вихрь Рэнкина (\citep{EmanuelRotunno1989,YanaseNiino2007}) или его модификации (см. \ref{app:B}), либо аномалия потенциального вихря в виде гауссова распределения, либо температурная аномалия, заданная с помощью тригонометрических функций (\citep{EggerHoinka2010}). Аномалия температуры часто помещалась приподнятой над поверхностью и интерпретировалась как очаг выделения скрытого тепла конденсации в средней тропосфере (\citep{RT2003}). При исследовании динамики диабатического вихря Россби (например, \citep{MooreMontgomery2005}) возмущение задавалось вблизи земной поверхности, которое с использованием принципа обратимости квазигеострофического потенциального вихря  позволяло получить сбалансированные поля скорости ветра и геопотенциала.
В виду того, что данная работа посвящена начальным стадиям развития мезомаштабных вихрей, в роли “затравочного” возмущения атмосферы выступала аномалия потенциальной температуры вблизи поверхности воды. Другими словами, фокус делается на зарождение полярного мезоциклона, триггером для которого служит выделение конечного количества тепла в нижних слоях тропосферы. В реальной атмосфере высоких широт такие аномалии могут возникать, например, при прохождении воздушного потока над неоднородном распределении льда и воды.
Каждый эксперимент начинался с инициализации приземной аномалии потенциальной температуры воздуха, расположенной в центре области. Аномалия имела куполообразную осесимметричную форму, заданную в виде произведения двух косинусов:
\begin{equation}\label{eq:initanom}
\theta'=\theta_{max}cos^2\left(\frac{\pi}{2}\frac{h}{H}\right) cos^2\left(\frac{\pi}{2}\frac{r}{r_{out}}\right),
\end{equation}
где $\theta_{max}$ - максимальная амплитуда, $r$ - радиус, $h$ - высота, $H$ - высота затухания аномалии, $r_out$ - радиус затухания аномалии.
В контрольном эксперименте значения параметров аномалии составляли: $\theta_{max}=5~K$, $r_{out}=50$ км, $H=1000$ м. Из формулы \ref{eq:initanom} видно, что амплитуда максимальна при $r=0$ и быстро падает с увеличением радиуса.

Наложение аномалии происходило через час модельного времени после инициализации фоновых полей, причем добавка искусственного возмущения совершалась не моментально, а в течение часа модельного времени. За счет постепенного возникновения аномалии достигалась устойчивость экспериментов и сглаженность полей. 
Описанный источник тепла служил лишь спусковым механизмом для дальнейшего роста вихревого возмущения. Иначе говоря, положительная аномалия температуры помещалась в область лишь в начале эксперимента и не поддерживалась в течение времени.

\subsection{План оценочных экспериментов}
\label{sec:expplan}
Кроме описанного выше контрольного случая были проведены несколько серий численных экспериментов с целью сопоставить влияние различных факторов на развитие возмущения в атмосфере. Список серий экспериментов и сравнение их настроек с контрольным приведен в табл. \ref{tab:expplan}, выступащей в качестве плана исследования, а полный перечень экспериментов представлен в приложении \ref{app:C}.

Ввиду поставленной цели – определения роли фоновых характеристик атмосферы в развитии полярного вихря – в процессе численного моделирования варьировались такие параметры, как статическая устойчивость атмосферы, температура воздуха и воды, содержание водяного пара в атмосфере и выделение скрытого тепла, скорость фонового потока. 

В разделе \ref{sec:theory} показано, что одним из важнейших источников энергии для полярных мезоциклонов являются потоки явного и скрытого тепла с поверхности. На интенсивность вихря, кроме того, влияет сила трения в пограничном слое атмосферы, то есть поток импульса. Поэтому в ряде экспериментов искусственно изменялись коэффициенты турбулентности или же потоки с поверхности “выключались” совсем.
Перенос тепла от поверхности в атмосферу зависит от параметризации турбулентной диффузии, или подсеточного перемешивания. Чувствительность к тому или иному турбулентному замыканию была выявлена в дополнительной серии экспериментов. 

\subsubsection{Чувствительность к амплитуде начальной аномалии}
Как видно из формулы \ref{eq:initanom}, характер температурной аномалии можно изменять через три параметра: максимум температуры в центре ($\theta_{max}$), высоту затухания аномалии ($H$) и радиус затухания аномалии ($r_{out}$). Значения перечисленных параметров варьировались в пределах от $2$ до $10\K$, от $1000$ до $2500\m$, от $25$ до $100\km$ соответственно.

\subsubsection{Чувствительность к разности температуры воды и воздуха}
Для выявления зависимости скорости роста полярного мезоциклона от соотношения температуры поверхности моря и температуры воздуха были проведены эксперименты, в которых изменялась или температура поверхности или температура атмосферы (при этом вертикальный градиент температуры во всей толще атмосферы не менялся). При этом был поставлен вопрос, влияет ли температура поверхности сама по себе (по аналогии с тропическими циклонами) или важна только разность между температурой воздуха и воды. Последняя менялась от $271$ до $283\K$, а температура нижних слоев тропосферы – от $251$ до $271\K$. Таким образом, разность температуры воздуха и воды варьировалась в пределах $12-32\K$.

\subsubsection{Чувствительность к стратификации атмосферы}
Стратификация атмосферы оказывает существенное влияние на рост вихревого возмущения в атмосферы: масштаб приспособления полей давления и скорости к термической аномалии обратно пропорционален статической устойчивости атмосферы. В проведенных экспериментах устойчивость атмосферы, выраженная через частоту плавучести (частоту Брента-Вяйсяля), изменялась от $0.0027$ до $0.0160\pers$. Этим значениям частоты плавучести соответствуют градиенты температуры от $0.2$ до $8\Kpkm$. Пониженная устойчивость в наших экспериментах по сравнению со средними значениями для климатологии полярных мезоциклонов оправдана, если рассматривать идеализированные эксперименты с точки зрения менее статически устойчивого конвективного пограничного слоя, формирующегося над водой при холодных вторжениях. 

\subsubsection{Чувствительность к процессам конденсации и количеству влаги}
Вклад процессов конвекции и конденсации в динамику полярных мезомасштабных циклонов был предметом многих исследований для различных регионов высоких широт и для вихрей различной структуры (например, \citep{SardieWarner1983, ForeEtAl2012}). В них убедительно показывается зависимость скорости роста вихря от выделения скрытого тепла конденсации. Зависимость интенсивности вихря от наличия влаги в атмосфере особенно просто объясняется с точки зрения условной неустойчивости второго рода, но это верно и для бароклинных волн (\citep{YanaseNiino2007}). В нашем исследовании контрольный эксперимент является “сухим”, то без выделения скрытого тепла конденсации в атмосфере. Для тестирования устойчивости начального возмущения температуры в условиях “влажной” атмосферы были проведены несколько экспериментов с наличием конденсации. При этом были проверены две параметризации, которые отличались формулой для давления насыщения водяного пара: для жидкой воды и для льда. Как уже было сказано, для простоты изменение влажности с высотой было линейным, и в контрольном эксперименте приводный максимум относительной влажности составлял $80\%$. В оценочных экспериментах это значение изменялось в пределах $60-90\%$.

\subsubsection{Чувствительность к фоновому потоку}
В действительности атмосфера в районах зарождения полярных мезоциклонов редко находится в состоянии покоя. Так, данные наблюдений (\citep{ForbesLottes1985}) свидетельствуют о том, что в среднем скорость ветра в слое $1000-500\hpa$ для случаев развивающихся вихрей составляет $7.1 \pm 4.3\mps$. Влияние скорости фонового потока на структуру вихря было рассмотрено на примере серии экспериментов, где ветер представлялся в виде зонального потока со скоростью $2-10\mps$. Так как при этом ветер не имел горизонтальных или вертикальных сдвигов, нельзя ожидать, что источником энергии для вихря будет баротропная или бароклинная неустойчивость атмосферы. Иными словами, можно предположить, что при наличии фонового потока поменяется лишь пространственная структура вихря из-за вклада адвективных слагаемых.

\subsubsection{Чувствительность к потокам тепла на поверхности}
Для лучшего понимания вклада потоков тепла и количества движения были сделаны дополнительные эксперименты. При этом потоки тепла либо отключались вовсе, либо искусственно модулировались при неизменных прочих условиях. 
Известно, что существует положительная обратная связь между величиной потока тепла и интенсивностью вихря, а именно скорости ветра в нем. Данный механизм играет значительную роль в тех вихрях, которые имеют конвективную природу и развиваются благодаря WISHE. Другой механизм, предложенный  для частичного объяснения природы полярных мезоциклонов -- CISK -- существенно зависит от работы силы трения, то есть от потока импульса с поверхности. Тот или другой механизм может доминировать в разных полярных вихрях, и необходимо было выяснить, какой из них наиболее важен в наших экспериментах. Этого можно добиться, по отдельности изменяя коэффициенты турбулентного обмена, стоящие в параметризациях потока тепла и потока импульса. В проведенных экспериментах эти коэффициенты были уменьшены или увеличены на $20-50\%$.

Помимо описанных экспериментов, были протестированы четыре различные параметризации турбулентных потоков. Для контрольного эксперимента использовалась схема Льюиса, описанная в работе \ref{Louis1979}, для экспериментов с разными коэффициентами турбулентности была использована параметризация Бусингера-Дайера (\citep{MirandaPhD}), показавшая близкое сходство с первой схемой. Кроме того, были получены результаты для параметризации, взятой из модели FLAKE (\citep{Mironov2006}), и для параметризации Зилитинкевича и Эзау.

\subsubsection{Чувствительность к подсеточному перемешиванию}
Большой интерес представляет влияние параметризации подсеточных процессов обмена теплом и импульсом в региональных численных моделях. Проблема правильного подбора схемы подсеточного перемешивания актуальна в виду как невысокого разрешения модели в данной работе, так и в оперативной практике, где шаг сетки модели все еще далек от точного воспроизведения конвекции в пограничном слое. Еще острее эта проблема стоит для глобальных моделей общей циркуляции атмосферы и для моделей климата. 

Исходя из этого была предпринята попытка сравнить две схемы параметризации подсеточной турбулентности, а именно замыкание Смагоринского-Лилли (\citep{MirandaPhD}) и схему нелокального замыкания (\citep{LupkesSchluenzen1996,NohEtAl2003}), добавленную для этого в модель ReMeDy.



\begin{thebibliography}{12}
 \bibitem{ER89}
Emanuel KA, Rotunno R. 1989. Polar lows as Arctic hurricanes. \textit{Tellus} \textbf{41A:} 1-17.
\end{thebibliography}

\end{document}