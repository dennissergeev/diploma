\documentclass[12pt,a4paper]{report}

%
% PACKAGES AND STYLES
%

% 
% Language
%
\usepackage[english]{babel}

%
% Paper size and margins
%

%
% References
%
\usepackage{natbib}

\bibliographystyle{plainnat}

\usepackage{showkeys} % show labels

\begin{document}
\setcounter{chapter}{3}
\chapter{Численные эксперименты}
\section{Мезомасштабная модель ReMeDy}
\subsection{Система уравнений (вывод?)}
\subsection{Усовершенствования в модели}
\begin{sqlist}
\item Периодические гр. усл.
\item Инициализация модели
\item Существующие проблемы использования модели
\end{sqlist}

\subsection{Конечно-разностная сетка модели}

\section{Постановка экспериментов}
Данное исследование основывается на идеализированных численных экспериментах, позволяющих улучшить наше понимание динамики полярных мезоциклонов. Преимущества идеализированных экспериментов были приведены в первом разделе, а здесь обосновывается постановка задачи применительно к генерации мезоциклонов.
На основе предшествующих работ, как теоретических, так и посвященных анализу данных наблюдений за полярными вихрями, параметры атмосферы в модели были заданы относительно близкими к воздушным массам морских областей высоких широт. Зимой в этих районах нередко наблюдаются холодные вторжения со льда на поверхность открытой воды, в результате чего разность температуры воздуха и воды  достигает десятков градусов.

Этап проведения экспериментов был начат с подбора вертикальных профилей температуры, скорости ветра и удельной влажности воздуха. Затем какой-либо один ключевой параметр в начальных условиях варьировался, а остальные параметры оставались равными «контрольным». На этом принципе построены остальные эксперименты, называемые далее оценочными.

После получения результатов в виде трехмерных и двумерных полей метеовеличин, а также таких интегральных показателей, как кинетическая энергия, планировалось сопоставить скорости роста возмущений, барическую тенденцию в центре вихря, временной ход максимальной скорости ветра и завихренности, а также другие характеристики эволюции мезоциклонов. Помимо этого, была проведена энергетическая диагностика мезомасштабных движений в области расчетов, о разработке которой рассказано ниже (раздел ???).

Расчетная область имела размеры  по горизонтали  $1000\km \times$ 
%$ 1000~\text{км}$ и $10.5~\text{км}$ 
%по вертикали (высота верхней границы области оправдана для моделирования тропосферных процессов приполярных широт). В нескольких сериях экспериментов по техническим причинам горизонтальные размеры области интегрирования составили $380~\text{км} \times 380~\text{км}$ и $1200~\text{км} \times 1200~\text{км}$, однако это почти не повлияло на динамику экспериментов.


\end{document}